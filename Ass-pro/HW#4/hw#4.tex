\documentclass{article}
\usepackage{amsmath}
\usepackage{amssymb}
\usepackage{enumitem}
\usepackage{listings}

\title{CS 5720 Design and Analysis of Algorithms \\ Homework \#4}
\author{Raja Kantheti}

\begin{document}
\maketitle

\section*{Question 1: Solving Recurrence Relations Using Master Theorem}

\begin{enumerate}[label=(\alph*)]
    \item \( T(n) = 5T(n/3) + n \)
    \begin{itemize}
        \item This recurrence fits the form \( T(n) = aT(n/b) + f(n) \), where \( a = 5 \), \( b = 3 \), and \( f(n) = n \).
        \item Calculate \( \log_b{a} = \log_3{5} \approx 1.46497 \).
        \item Compare \( f(n) = n \) with \( n^{\log_b{a}} \).
        \item Since \( f(n) = O(n^{\log_b{a} - \epsilon}) \) for some \(\epsilon > 0\) \\(specifically, \( n^1 \) vs. \( n^{1.46497} \)), by the Master Theorem (Case 1):
        \item \( T(n) = \Theta(n^{\log_3{5}}) \approx \Theta(n^{1.46497}) \)
    \end{itemize}

    \item \( T(n) = 2.7T(n/5) + n^2 \)
    \begin{itemize}
        \item This recurrence fits the form \( T(n) = aT(n/b) + f(n) \), where \( a = 2.7 \), \( b = 5 \), and \( f(n) = n^2 \).
        \item Calculate \( \log_b{a} = \log_5{2.7} \approx 0.58975 \).
        \item Compare \( f(n) = n^2 \) with \( n^{\log_b{a}} \).
        \item Since \( f(n) = \Omega(n^{\log_b{a} + \epsilon}) \) for some \(\epsilon > 0\) (specifically, \( n^2 \) vs. \( n^{0.58975} \)), by the Master Theorem (Case 3):
        \item \( T(n) = \Theta(n^2) \)
    \end{itemize}

    \item \( T(n) = 2T(n-1) + n \)
    \begin{itemize}
        \item This recurrence does not fit the Master Theorem directly because \( T(n) \) is based on \( T(n-1) \), not \( T(n/b) \).
        \item However, this can be solved by iteration or substitution. This is a linear homogeneous recurrence with additional term:
        \item \( T(n) = 2T(n-1) + n \)
        \item Using iteration:
        \begin{align*}
            T(n) &= 2(2T(n-2) + (n-1)) + n = 4T(n-2) + 2(n-1) + n \\
            T(n) &= 4(2T(n-3) + (n-2)) + 2(n-1) + n = 8T(n-3) + 4(n-2) + 2(n-1) + n \\
        \end{align*}
        \item Continuing this pattern and summing the series, we get:
        \item \( T(n) = 2^n T(0) + \sum_{k=0}^{n-1} 2^k k \)
        \item \( T(n) = 2^n + n \cdot 2^n \)
        \item The dominant term is \( 2^n \), so:
        \item \( T(n) = \Theta(2^n) \)
    \end{itemize}

    \item \( T(n) = 1.1T(0.2n) + 1 \)
    \begin{itemize}
        \item This recurrence fits the form \( T(n) = aT(n/b) + f(n) \), where \( a = 1.1 \), \( b = 0.2 \), and \( f(n) = 1 \).
        \item Calculate \( \log_{0.2}{1.1} \).
        \item Since \( b < 1 \), this doesn't fit the typical Master Theorem format directly, making it difficult to apply the theorem directly.
    \end{itemize}

    \item \( T(n) = 2T(n/2) + n \log_2 n \)
    \begin{itemize}
        \item This recurrence fits the form \( T(n) = aT(n/b) + f(n) \), where \( a = 2 \), \( b = 2 \), and \( f(n) = n \log_2 n \).
        \item Calculate \( \log_b{a} = \log_2{2} = 1 \).
        \item Compare \( f(n) = n \log_2 n \) with \( n^{\log_b{a}} = n^1 \).
        \item Since \( f(n) = \Theta(n \log n) \), which is polynomially larger than \( n^1 \):
        \item \( T(n) = \Theta(n \log^2 n) \)
    \end{itemize}

    \item \( T(n) = 2T(n/2) + \sqrt{n} \)
    \begin{itemize}
        \item This recurrence fits the form \( T(n) = aT(n/b) + f(n) \), where \( a = 2 \), \( b = 2 \), and \( f(n) = \sqrt{n} \).
        \item Calculate \( \log_b{a} = \log_2{2} = 1 \).
        \item Compare \( f(n) = \sqrt{n} \) with \( n^{\log_b{a}} = n^1 \).
        \item Since \( f(n) = O(n^{1-\epsilon}) \):
        \item \( T(n) = \Theta(n) \)
    \end{itemize}

    \item \( T(n) = 4T(n/2) + \sqrt{n^4 - n + 10} \)
    \begin{itemize}
        \item This recurrence fits the form \( T(n) = aT(n/b) + f(n) \), where \( a = 4 \), \( b = 2 \), and \( f(n) = \sqrt{n^4 - n + 10} \).
        \item Calculate \( \log_b{a} = \log_2{4} = 2 \).
        \item Compares \( f(n) = \sqrt{n^4} = n^2 \) with \( n^{\log_b{a}} = n^2 \).
        \item Since \( f(n) = \Theta(n^2) \):
        \item \( T(n) = \Theta(n^2 \log n) \)
    \end{itemize}

    \item \( T(n) = 7T(n/3) + \sum_{i=1}^n i \)
    \begin{itemize}
        \item This recurrence fits the form \( T(n) = aT(n/b) + f(n) \), where \( a = 7 \), \( b = 3 \), and \( f(n) = \sum_{i=1}^n i = \frac{n(n+1)}{2} = \Theta(n^2) \).
        \item Calculate \( \log_b{a} = \log_3{7} \approx 1.77124 \).
        \item Compare \( f(n) = n^2 \) with \( n^{\log_b{a}} \).
        \item Since \( f(n) = \Theta(n^2) \), and \( n^2 > n^{\log_b{7}} \):
        \item \( T(n) = \Theta(n^2) \)
    \end{itemize}

    \item \( T(n) = 4T(n/2) + n^n \)
    \begin{itemize}
        \item This recurrence cannot be solved by the Master Theorem because the function \( f(n) = n^n \) is super-polynomial and does not fit the form required for the Master Theorem application.
    \end{itemize}

    \item \( T(n) = 8T(n/3) + n^3 \)
    \begin{itemize}
        \item This recurrence fits the form \( T(n) = aT(n/b) + f(n) \), where \( a = 8 \), \( b = 3 \), and \( f(n) = n^3 \).
        \item Calculate \( \log_b{a} = \log_3{8} \approx 1.89279 \).
        \item Compare \( f(n) = n^3 \) with \( n^{\log_b{a}} \).
        \item Since \( f(n) = \Omega(n^{\log_b{a} + \epsilon}) \):
        \item \( T(n) = \Theta(n^3) \)
    \end{itemize}
\end{enumerate}

\section*{Question 2: Divide-and-Conquer Algorithms}

\subsection*{(a) Algorithm 1}
Need to check this agian. 
\begin{itemize}
    \item \textbf{Worst-Case Order of Growth}:
    \begin{itemize}
        \item \( T(n) = T(n/3) + O(1) \)
        \item Using Master Theorem: \( a = 1, b = 3, f(n) = O(1) \)
        \item Since \( f(n) = O(n^c) \) for \( c = 0 \), and \( c < \log_b{a} \) (i.e., 0 < 0), case 1 applies.
        \item Thus, \( T(n) = \Theta(\log n) \).
    \end{itemize}
    
    \item \textbf{Worst-Case Input}:
    \begin{itemize}
        \item An input where Rec1 always takes the second recursive call, causing maximum depth recursion.
    \end{itemize}

    \item \textbf{Best-Case Complexity}:
    \begin{itemize}
        \item The best-case occurs when the input array has only one element (n = 1), and the algorithm returns A[0] immediately.
        \item then the complexity would be  O(1)
    \end{itemize}

    \item \textbf{Best-Case Input}:
    \begin{itemize}
        \item  input array has only one element (n = 1). 
    \end{itemize}
\end{itemize}

\subsection*{(b) Algorithm 2}

\begin{itemize}
    \item \textbf{Worst-Case Order of Growth}:
    \begin{itemize}
        \item \( T(n) = 2T(n/2) + O(1) \)
        \item Using Master Theorem: \( a = 2, b = 2, f(n) = O(1) \)
        \item Since \( f(n) = O(n^c) \) for \( c = 0 \), and \( c < \log_b{a} \) (i.e., 0 < 1), case 1 applies.
        \item Thus, \( T(n) = \Theta(n) \).
    \end{itemize}
    
    \item \textbf{Worst-Case Input}:
    \begin{itemize}
        \item Any input of length n, as the algorithm splits the array and recurses into both halves.
    \end{itemize}

    \item \textbf{Best-Case Complexity}:
    \begin{itemize}
        \item The best-case occurs when the input array has only one element (n = 1), and the algorithm returns A[0] immediately.
        \item then the complexity would be  O(1)
    \end{itemize}

    \item \textbf{Best-Case Input}:
    \begin{itemize}
        \item  input array has only one element (n = 1).
    \end{itemize}
\end{itemize}

\end{document}
