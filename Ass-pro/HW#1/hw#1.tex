\documentclass{article}
\usepackage{amsmath}
\usepackage{amsfonts}
\usepackage{amssymb}
\begin{document}

\title{CS 5720 Design and Analysis of Algorithms \\ Homework \#1}
\author{Raja Kantheti. }
\maketitle

\section*{Problem 1}
Determine the order of growth \(\Theta(\cdot)\) for each of the following functions:

\paragraph{Generalized justification:}

The goal is to determine which term in the function/eqution has the deciding power in how the function grows, it is the largest term in the function/equwation that has the deciding power.

\subsection*{(a) \((n^2 + 1)^{10}\)}
\textbf{Justification:} The dominant term in the expression is \( (n^2)^{10} = n^{20} \). According to polynomial theorem, the order of growth is affected most by $n^{20}$.
\\ $\therefore$\textbf{ Order of growth:} \( \Theta(n^{20}) \)

\subsection*{(b) \(\sqrt{10n^2 + 7n + 3}\)}
\textbf{Justification:} The term \( 10n^2 \) inside the square root dominates as \( n \) grows, thus \( \sqrt{10n^2} = n \sqrt{10} \approx n \). Lower order terms are insignificant.
\\ $\therefore$\textbf{ Order of growth:} \( \Theta(n) \)

\subsection*{(c) \(2n \log((n+2)^2) + (n+2)^2 \log(n/2)\)}
\textbf{Justification:} After simplifying, \( 2n \log((n+2)^2) = 4n \log(n+2) \approx 4n \log n \) and \((n+2)^2 \log(n/2) \approx n^2 \log n \). The \( n^2 \log n \) term dominates.
\\ \textbf{Order of growth:} \( \Theta(n^2 \log n) \)

\subsection*{(d) \(2^{n+1} + 3^{n-1}\)}
\textbf{Justification:} Between \( 2^{n+1} \approx 2^n \) and \( 3^{n-1} \approx 3^n \), \( 3^n \) grows faster, hence it dominates.
\\ \textbf{Order of growth:} \( \Theta(3^n) \)

\subsection*{(e) \(\lfloor \log_2 n \rfloor\)}
\textbf{Justification:} The floor function does not change the order of growth, this only depends on how the logarithm is scaling.
\\ \textbf{Order of growth:} \( \Theta(\log n) \)

\section*{Problem 2}
Prove the following assertions by using the definitions of the notations involved, or disprove them by giving a specific counterexample.

\subsection*{(a) If \(t(n) \in O(g(n))\), then \(g(n) \in \Omega(t(n))\).}
\textbf{Proof:} 
By definition of Big-O, \( t(n) \in O(g(n)) \) implies \(\exists\) a constant \( c > 0 \) and \( n_0 \) such that 
\[
t(n) \leq c \cdot g(n) \quad \text{for all} \quad n \geq n_0.
\]
By definition of Big-Omega, \( g(n) \in \Omega(t(n)) \) implies \(\exists\) a constant \( c' > 0 \) and \( n_0' \) such that 
\[
g(n) \geq c' \cdot t(n) \quad \text{for all} \quad n \geq n_0'.
\]
Since \( t(n) \leq c \cdot g(n) \), we can rewrite this as \( g(n) \geq \frac{1}{c} \cdot t(n) \). Therefore, the statement \( t(n) \in O(g(n)) \implies g(n) \in \Omega(t(n)) \) is true.

\subsection*{(b) \(\Theta(\alpha g(n)) = \Theta(g(n))\), where \(\alpha > 0\).}
\textbf{Proof:} 
Since \( \alpha \) is a positive constant, multiplying by \(\alpha\) does not change the order of growth. By definition of Big-Theta, \( f(n) = \Theta(g(n)) \) if and only if \( f(n) \) is both \( O(g(n)) \) and \( \Omega(g(n)) \). Thus,
\[
\alpha g(n) \in O(g(n)) \quad \text{and} \quad \alpha g(n) \in \Omega(g(n)).
\]
This implies:
\[
\Theta(\alpha g(n)) = \Theta(g(n)).
\]
Therefore, the statement is true.

\subsection*{(c) \(\Theta(g(n)) = O(g(n)) \cap \Omega(g(n))\).}
\textbf{Proof:} 
By definition, \( \Theta(g(n)) \) means the function is bounded both above and below by \( g(n) \) up to constant factors. Specifically,
\[
\Theta(g(n)) = \{ f(n) : \exists c_1, c_2 > 0, \text{ and } n_0 \text{ such that } c_1 g(n) \leq f(n) \leq c_2 g(n) \text{ for all } n \geq n_0 \}.
\]
This is equivalent to the intersection of \( O(g(n)) \) (functions asymptotically upper-bounded by \( g(n) \)) and \( \Omega(g(n)) \) (functions asymptotically lower-bounded by \( g(n) \)). Hence,
\[
\Theta(g(n)) = O(g(n)) \cap \Omega(g(n)).
\]
Therefore, the statement is true.

\subsection*{(d) For any two nonnegative functions \( t(n) \) and \( g(n) \) defined on the set of nonnegative integers, either \( t(n) \in O(g(n)) \) or \( t(n) \in \Omega(g(n)) \), or both.}
\textbf{Proof:} 
For any two functions, one will eventually grow faster or they will be proportional. Specifically, given nonnegative functions \( t(n) \) and \( g(n) \), either:
\[
\exists c > 0 \text{ and } n_0 \text{ such that } t(n) \leq c \cdot g(n) \quad \text{for all} \quad n \geq n_0,
\]
implying \( t(n) \in O(g(n)) \), or:
\[
\exists c' > 0 \text{ and } n_0' \text{ such that } g(n) \leq c' \cdot t(n) \quad \text{for all} \quad n \geq n_0',
\]
implying \( t(n) \in \Omega(g(n)) \). If both conditions hold, then \( t(n) \) and \( g(n) \) are asymptotically proportional, and both \( t(n) \in O(g(n)) \) and \( t(n) \in \Omega(g(n)) \). Therefore, the statement is true.

\section*{Problem 3}
Determine the order of growth (\(\Theta(\cdot)\)) for each of the following functions. Show your work.

\paragraph{Generalized justification:} It is the same as problem one, but we have to understannd how the \(\sum\) unrolls.
 
\subsection*{(a) \( T(n) = \sum_{i=1}^{2n} i \)}
\textbf{Justification:} Sum of the first \(2n\) natural numbers is \( \frac{2n(2n+1)}{2} = 2n^2 + n \). The dominant term is \( 2n^2 \).
\\ \textbf{Order of growth:} \( \Theta(n^2) \)

\subsection*{(b) \( T(n) = \sum_{i=1}^{n} \sum_{j=i}^{n} n \)}
\textbf{Justification:} The inner sum is \((n-i+1)n\). Summing from \(i=1\) to \(n\), we get \( n \cdot \frac{n(n+1)}{2} = \frac{n^3 + n^2}{2} \). The dominant term is \( n^3 \).
\\ \textbf{Order of growth:} \( \Theta(n^3) \)

\subsection*{(c) \( T(n) = \sum_{i=1}^{n} n^2 \)}
\textbf{Justification:} Summing \( n^2 \) for \( n \) times gives \( n \cdot n^2 = n^3 \).
\\ \textbf{Order of growth:} \( \Theta(n^3) \)

\subsection*{(d) \( T(n) = \sum_{i=1}^{n^2} i \)}
\textbf{Justificaton:} Sum of the first \(n^2\) natural numbers is \( \frac{n^2(n^2 + 1)}{2} \approx n^4 \).
\\ \textbf{Order of growth:} \( \Theta(n^4) \)

\subsection*{(e) \( T(n) = \sum_{i=1}^{n} \sum_{j=1}^{n} \sum_{k=1}^{n} \sum_{\ell=1}^{n} \ell \)}
\textbf{Justification:} The innermost sum is \( \sum_{\ell=1}^{n} \ell = \frac{n(n+1)}{2} \approx n^2 \). Repeatedly summing \( n^2 \) over the remaining loops gives \( n \cdot n \cdot n \cdot n^2 = n^5 \).
\\ \textbf{Order of growth:} \( \Theta(n^5) \)

\end{document}
