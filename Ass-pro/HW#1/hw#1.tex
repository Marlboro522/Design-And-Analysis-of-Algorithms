% \documentclass[a4paper,12pt]{article}
% \usepackage{setspace}

% \usepackage{amssymb}

% \begin{document}

%     \title{Homework \#1}
%     \author{Raja Kantheti}
%     \date{\today}
%     \maketitle
    
%     \pagenumbering{arabic}
    
%     \section{\underline{\underline{Problem1}}}
%     \subsection{\underline{a}}
%     Given equation: $(n^2 + 10)^{10}$.\\
%     By Max Theorem, \emph{max($(n^2)^{10}, 10 ^{10})$} is $(n^2)^{10}$
%     Hence it should be in the order of growth, $\Theta(n^{20})$.
%     \subsection{\underline{b}}
%     Given equation: $\sqrt{10n^2+7n+3}$.

%     \subsection[short]{\underline{c}}
%     Given equation: $2n\log\left((n+2)^2\right) + (n+2)^2\log(n/2)$

%     \subsection[short]{\underline{d}}
%     Given equation: $2^{n+1} + 3^{n-1}$

%     \subsection[short]{\underline{e}}
%     Given equation: $\lfloor{\log_2n}\rfloor$.

%     \section{\underline{\underline{Problem2}}}

%     \section{\underline{\underline{Problem3}}}
    
%     \section{\underline{\underline{Problem4}}}
    
% \end{document}

\documentclass{article}
\usepackage{amsmath}
\usepackage{amsfonts}
\usepackage{amssymb}
\begin{document}

\title{CS 5720 Design and Analysis of Algorithms \\ Homework \#1}
\author{Raja Kantheti}
\maketitle

\section*{Problem 1}
Determine the order of growth \(\Theta(\cdot)\) for each of the following functions:

\subsection*{(a) \((n^2 + 1)^{10}\)}
\textbf{Dominant term:} \( (n^2)^{10} = n^{20} \) \\
\textbf{Order of growth:} \( \Theta(n^{20}) \)

\subsection*{(b) \(\sqrt{10n^2 + 7n + 3}\)}
\textbf{Dominant term:} \(\sqrt{10n^2} = n \sqrt{10} \approx n \) \\
\textbf{Order of growth:} \( \Theta(n) \)

\subsection*{(c) \(2n \log((n+2)^2) + (n+2)^2 \log(n/2)\)}
\textbf{First term analysis:} 
\begin{align*}
2n \log((n+2)^2) &= 2n \cdot 2 \log(n+2) \\
&= 4n \log(n+2)
\end{align*}
\textbf{Dominant term:} \( 4n \log n \) \\

\textbf{Second term analysis:} 
\begin{align*}
(n+2)^2 \log(n/2) &= (n^2 + 4n + 4) \log(n/2) \\
&\approx n^2 \log n
\end{align*}
\textbf{Dominant term:} \( n^2 \log n \) \\

\textbf{Overall dominant term:} \( n^2 \log n \) \\
\textbf{Order of growth:} \( \Theta(n^2 \log n) \)

\subsection*{(d) \(2^{n+1} + 3^{n-1}\)}
\textbf{Dominant term:} \(2^{n+1} \approx 2^n \) and \(3^{n-1} \approx 3^n \) \\
Since \( 3^n \) grows faster than \( 2^n \), \\
\textbf{Order of growth:} \( \Theta(3^n) \)

\subsection*{(e) \(\lfloor \log_2 n \rfloor\)}
\textbf{Order of growth:} \( \Theta(\log n) \)

\section*{Problem 2}
Prove the following assertions by using the definitions of the notations involved, or disprove them by giving a specific counterexample.

\subsection*{(a) If \(t(n) \in O(g(n))\), then \(g(n) \in \Omega(t(n))\).}
\textbf{Proof:} \\
By definition of Big-O, \( t(n) \in O(g(n)) \) means there exist constants \( c > 0 \) and \( n_0 \) such that \( t(n) \leq c \cdot g(n) \) for all \( n \geq n_0 \). \\
By definition of Big-Omega, \( g(n) \in \Omega(t(n)) \) means there exist constants \( c' > 0 \) and \( n_0' \) such that \( g(n) \geq c' \cdot t(n) \) for all \( n \geq n_0' \). \\
Hence, if \( t(n) \leq c \cdot g(n) \), then \( g(n) \geq \frac{1}{c} \cdot t(n) \). \\
Therefore, the statement is true.

\subsection*{(b) \(\Theta(\alpha g(n)) = \Theta(g(n))\), where \(\alpha > 0\).}
\textbf{Proof:} \\
By definition of Theta, \( f(n) = \Theta(g(n)) \) if \( f(n) \) is both \( O(g(n)) \) and \( \Omega(g(n)) \). \\
Given \( \alpha > 0 \), \( \alpha g(n) \) is simply a constant multiple of \( g(n) \). \\
Constants do not affect the growth rate, thus \( \Theta(\alpha g(n)) = \Theta(g(n)) \). \\
Therefore, the statement is true.

\subsection*{(c) \(\Theta(g(n)) = O(g(n)) \cap \Omega(g(n))\).}
\textbf{Proof:} \\
By definition, \( \Theta(g(n)) \) means the function is bounded both above and below by \( g(n) \) up to constant factors. \\
\( O(g(n)) \) defines an upper bound, and \( \Omega(g(n)) \) defines a lower bound. \\
Intersection of these two sets means the function is both bounded above and below by \( g(n) \), which is the definition of \( \Theta(g(n)) \). \\
Therefore, the statement is true.

\subsection*{(d) For any two nonnegative functions \( t(n) \) and \( g(n) \) defined on the set of nonnegative integers, either \( t(n) \in O(g(n)) \) or \( t(n) \in \Omega(g(n)) \), or both.}
\textbf{Proof/Disproof:} \\
Consider \( t(n) = n \) and \( g(n) = n \). \\
Both functions are equal, hence \( t(n) \in O(g(n)) \) and \( t(n) \in \Omega(g(n)) \). \\
Consider \( t(n) = n \) and \( g(n) = n^2 \). \\
\( t(n) \in O(g(n)) \) since \( n \leq n^2 \). \\
Consider \( t(n) = n^2 \) and \( g(n) = n \). \\
\( t(n) \in \Omega(g(n)) \) since \( n^2 \geq n \). \\
Therefore, the statement is true as one of these conditions will always hold.

\section*{Problem 3}
Determine the order of growth (\(\Theta(\cdot)\)) for each of the following functions. Show your work.

\subsection*{(a) \( T(n) = \sum_{i=1}^{2n} i \)}
\textbf{Sum of first \(2n\) natural numbers:}
\[
\frac{2n(2n+1)}{2} = 2n^2 + n
\]
\textbf{Order of growth:} \( \Theta(n^2) \)

\subsection*{(b) \( T(n) = \sum_{i=1}^{n} \sum_{j=i}^{n} n \)}
\textbf{Inner sum analysis:}
\[
\sum_{j=i}^{n} n = (n-i+1)n
\]
\textbf{Outer sum analysis:}
\[
\sum_{i=1}^{n} (n-i+1)n = n \sum_{i=1}^{n} (n-i+1) = n \sum_{k=1}^{n} k = n \cdot \frac{n(n+1)}{2}
\]
\textbf{Order of growth:} \( \Theta(n^3) \)

\subsection*{(c) \( T(n) = \sum_{i=1}^{n} n^2 \)}
\textbf{Sum analysis:}
\[
n \cdot n^2 = n^3
\]
\textbf{Order of growth:} \( \Theta(n^3) \)

\subsection*{(d) \( T(n) = \sum_{i=1}^{n^2} i \)}
\textbf{Sum of first \(n^2\) natural numbers:}
\[
\frac{n^2(n^2 + 1)}{2} \approx n^4
\]
\textbf{Order of growth:} \( \Theta(n^4) \)

\subsection*{(e) \( T(n) = \sum_{i=1}^{n} \sum_{j=1}^{n} \sum_{k=1}^{n} \sum_{\ell=1}^{n} \ell \)}
\textbf{Innermost sum:}
\[
\sum_{\ell=1}^{n} \ell = \frac{n(n+1)}{2} \approx n^2
\]
\textbf{Next sum:}
\[
\sum_{k=1}^{n} n^2 = n^3
\]
\textbf{Next sum:}
\[
\sum_{j=1}^{n} n^3 = n^4
\]
\textbf{Outermost sum:}
\[
\sum_{i=1}^{n} n^4 = n^5
\]
\textbf{Order of growth:} \( \Theta(n^5) \)

\end{document}
