\documentclass[a4paper,12pt]{article}
\usepackage{setspace}

\usepackage{amssymb}

\begin{document}

    \title{Homework \#0}
    \author{Raja Kantheti}
    \date{\today}
    \maketitle
    
    \pagenumbering{arabic}
    
    \section{\underline{\underline{Problem1}}}
    Determine whether all entries of array \emph{A} are distinct, that is, return \emph{True}
     if there are no duplicate entries in \emph{A}, and \emph{False} otherwise.
     \subsection{pseudocode}
     \begin{tabbing}
     seen = \{\} \# seen is a set, so no duplicate values.\\
     for i in \emph{0} to \emph{len(A)-1}:\\
     \hspace{1em}do:\\
     \hspace{2em}if a[i] in seen:\\
     \hspace{3em}return \emph{\textit{False}}\\
     \hspace{2em}else:\\
     \hspace{3em}seen.append(a[i])\\
     return \emph{\textit{True}}
     \end{tabbing}
     
     \subsection{Demonstration of Operation}
     \underline{Example1:}
     \\Example array \emph{A} = [1,2,2,3,4]
     \paragraph{}
     The For loop starts at index 0 and does two things, checks if the value is already in the set \emph{seen} and, if not, it adds the value at the current index to set \emph{seen}.
     As the loop progresses, the append operation is executed for the elements at indices 0,1 to the set \emph{seen}, which does not allow duplicates. 
     After reaching the index 2 of the Array \emph{A}, since the value at A[3], exists in the set
     \emph{seen}, the function returns \emph{False} and the loop is terminated.\\
     \underline{Example2:}\\
     Example array \emph{A} = [1,2,3,4,5]
     \paragraph{}
        The For loop starts at index 0 and appends the elements at indices 0,1,2,3,4 to the set \emph{seen}, which does not allow duplicates.
        The index is exhausted, which means the Array is traversed once and there are no duplicates found. So, True is returned.
     
        \section{\underline{\underline{Problem2}}}
    Compute $\lfloor\sqrt{n}\rfloor$ for any positive integer $n$.
    Besides assignment and comparison, your algorithm may use only the four basic
    arithmetical operations $(+,-,*,/)$.
    \subsection{pseudocode}
    \begin{tabbing}
        we can use a binary search for this.\\
    high = n\\
    low = 1\\
    while low $<$ high:\\
    \hspace{1em}mid = (low + high) / 2\\
    \hspace{1em}if mid * mid $=$ n:\\
    \hspace{2em}return mid\\
    \hspace{1em}else if mid * mid $<$ n:\\
    \hspace{2em}low = mid + 1\\
    \hspace{1em}else:\\
    \hspace{2em}high = mid - 1\\
    return high
    \end{tabbing}

    \subsection{Demonstration of Operation}
    \underline{Example1:}\\
    For n = 9,\\
    \textit{First Iteration:}\\
    low = 1\\
    high = 9\\
    mid = 5\\
    $5^2$ is 25, which is greater than 9. So, high is updated to 4.\\
    \textit{Second Iteration:}\\
    low = 1\\
    high = 4\\
    mid = 2\\
    $2^2$ is 4, which is less than 9. So, low is updated to 3.\\
    \textit{Third Iteration:}\\
    low = 3\\
    high = 4\\
    mid = 3\\
    Since $3^2$ is 9, it is returned.\\
    \underline{Example2:}\\
    For n=4\\
    \textit{First Iteration:}\\
    low =1\\
    high =4\\
    mid = 2\\
    $2^2$ is 4, which is equal to n. So, 2 is returned.


    \section{\underline{\underline{Problem3}}}
    Compute the product $mn$ for any positive integers $m$ and $n$, using
    only addition $(+)$.
    \subsection{pseudocode}
    \begin{tabbing}
        function product(n, m):\\
    \hspace{1em}result = 0\\
    \hspace{1em}for i in range(0, m):\\
    \hspace{2em}result += n\\
    \hspace{1em}return result
    \end{tabbing}
    \subsection{Demonstration of Operation}
    \underline{Example1:}\\
    For m = 3, n = 4,\\
    \textit{First Iteration:}\\
    result = 0\\
    i = 0\\
    result = 0 + 4 = 4\\
    \textit{Second Iteration:}\\
    result = 4\\
    i = 1\\
    result = 4 + 4 = 8\\
    \textit{Third Iteration:}\\
    result = 8\\
    i = 2\\
    result = 8 + 4 = 12\\
    12 is returned.\\
    \underline{Example2:}\\
    For m = 2, n = 5,\\
    \textit{First Iteration:}\\
    result = 0\\
    i = 0\\
    result = 0 + 5 = 5\\
    \textit{Second Iteration:}\\
    result = 5\\
    i = 1\\
    result = 5 + 5 = 10\\
    10 is returned.


    \section{\underline{\underline{Problem4}}}
    Compute the quantity $m^n$ for any positive integers $m$ and $n$, using
    only addition $(+)$ or your algorithm from Question 3.
    \subsection{pseudocode}
    \begin{tabbing}
        function power(m, n):\\
    \hspace{1em}result = 1\\
    \hspace{1em}for i in range(0, n):\\
    \hspace{2em}result = product(result, m)\\
    \hspace{1em}return result
    \end{tabbing}
    \subsection{Demonstration of Operation}
    \underline{Example1:}\\
    For m = 2, n = 3,\\
    \textit{First Iteration:}\\
    result = 1\\
    i = 0\\
    result = product(1, 2) = 2\\
    \textit{Second Iteration:}\\
    result = 2\\
    i = 1\\
    result = product(2, 2) = 4\\
    \textit{Third Iteration:}\\
    result = 4\\
    i = 2\\
    result = product(4, 2) = 8\\
    8 is returned.\\
    \underline{Example2:}\\
    For m = 3, n = 2,\\
    \textit{First Iteration:}\\
    result = 1\\
    i = 0\\
    result = product(1, 3) = 3\\
    \textit{Second Iteration:}\\
    result = 3\\
    i = 1\\
    result = product(3, 3) = 9\\
    9 is returned.
\end{document}

