\documentclass{article}
\usepackage{amsmath}
\usepackage{amssymb}
\usepackage{enumitem}
\usepackage{listings}
\usepackage[margin=2cm]{geometry} 

\title{CS 5720 Design and Analysis of Algorithms \\ Homework \#5}
\author{Your Name}
\date{\today}

\begin{document}
\maketitle

\section*{Question 1: MergeSort and QuickSort Analysis}
% Check everything again
\subsection*{(a) MergeSort on Sorted Arrays}
\subsubsection*{REcurrence Relation: }
      \[
      T(n) = 2T\left(\frac{n}{2}\right) + n - 1
      \]
   \begin{enumerate}
      \item Splitting the array takes $2T(n/2)$ comparisons.
      \item Merging two sorted arrays of size $n/2$ takes $n - 1$ comparisons (since each comparison merges one element).
   \end{enumerate}
\subsubsection*{Solution Using Master Theorem: }
   \begin{enumerate}
      \item Here, $a = 2$, $b = 2$, and $f(n) = n - 1 = \Theta(n)$.
      \item Calculate $\log_b{a} = \log_2{2} = 1$.
      \item Since $f(n) = \Theta(n)$, we use Case 2 of the Master Theorem.
   \end{enumerate}
   \[
   T(n) = \Theta(n \log n)
   \]

\subsection*{(b) MergeSort on Reverse-Sorted Arrays}
\subsubsection*{REcurrence Relation: }
   \[
   T(n) = 2T\left(\frac{n}{2}\right) + n - 1
   \]
   $\longrightarrow$ The recurrence relation is the same as for sorted arrays because MergeSort's comparison count does not depend on the order of the elements.
\subsubsection*{Solution Using Master Theorem: }
   $\longrightarrow$ Same as part (a).
   \[
   T(n) = \Theta(n \log n)
   \]

\subsection*{(c) QuickSort on Sorted Arrays}
\subsubsection*{Recurrence Relation:}
   \[
   T(n) = T(n-1) + n - 1
   \]
   \begin{enumerate}
      \item The first partition step compares each element to the pivot, making $n-1$ comparisons.
      \item The lazy pivot selection (first element in sorted array) leads to one partition of size $n-1$ and one partition of size 0.
   \end{enumerate}
\subsubsection*{Solution:}:
   \begin{enumerate}
      \item Unroll the recurrence:
      \[
      T(n) = T(n-1) + (n-1) = T(n-2) + (n-2) + (n-1) = \cdots = T(1) + \sum_{i=1}^{n-1} i = T(1) + \frac{n(n-1)}{2}
      \]
      \item Since $T(1) = 0$ (base case), we have:
      \[
      T(n) = \Theta(n^2)
      \]
   \end{enumerate}

\subsection*{(d) QuickSort on Arrays of All Equal Elements}
\subsubsection*{Recurrence Relation: }
\[
T(n) = 2T(n-1) + (n-1)
\]
   \begin{enumerate}
      \item Each partition step still compares each element to the pivot, making $n-1$ comparisons.
      \item For arrays with all equal elements, each partition splits the array into two equal parts.
   \end{enumerate}
\subsubsection*{Solution:  }
   \begin{enumerate}
      \item Unroll the recurrence:
      \[
      T(n) = T(n-1) + (n-1) = T(n-2) + (n-2) + (n-1) = \cdots = T(1) + \sum_{i=1}^{n-1} i = T(1) + \frac{n(n-1)}{2}
      \]
      \item Since $T(1) = 0$ (base case), we have:
      \[
      T(n) = \Theta(n^2)
      \]
   \end{enumerate}

\section*{Question 2: Optimizing QuickSort for Sorted Arrays}
We cannot choose the first or lasst elements of the arrays as one partition might contain most of the elements of the array and further down the recursion the resulting partitions will have one empty subarray and one subarray containing all the remaining elements, the depth of the
recursion tree will be at it's maximum yeilding a time complexity of $\Theta(n^2)$.
Choosign randomly might help in some scenarios but we can be certian that in some cases this will result in sub-optimal performance. 

To optimize QuickSort for sorted arrays, we should select the pivot such that the partitions are balanced.Since we know that the array is sorted.we just have to select the middle element of the array and make it our pivot.


\subsection*{Pivot Selection Rule}
- **Rule**: Select the median of the array as the pivot. In sorted arrays, this will divide the array into two equal parts.

\subsection*{Resulting Recurrence Relation}
   \[
   T(n) = 2T\left(\frac{n}{2}\right) + n - 1
   \]
   $\longrightarrow$Selecting the median as the pivot results in two partitions of size $n/2$.
\subsubsection*{Solution using Master Theorem: }
   \begin{enumerate}
      \item Here, $a = 2$, $b = 2$, and $f(n) = n - 1 = \Theta(n)$.
      \item Calculate $\log_b{a} = \log_2{2} = 1$.
      \item Since $f(n) = \Theta(n)$, we use Case 2 of the Master Theorem.
   \end{enumerate}
   \[ T(n) = \Theta(n \log n)\]

\end{document}
