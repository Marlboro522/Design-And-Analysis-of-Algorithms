\documentclass{article}
\usepackage{amsmath}
\usepackage{amssymb}
\usepackage{enumitem}
\usepackage{listings}
\usepackage[margin=2cm]{geometry} 

\title{CS 5720 Design and Analysis of Algorithms \\ Homework \#5}
\author{Your Name}
\date{\today}

\begin{document}
\maketitle

\section*{Question 1: MergeSort and QuickSort Analysis}
% Check everything again
For each part (a)-(d), we will provide a recurrence relation for the number of element comparisons required by the algorithm under the given conditions and solve the recurrence relation using the Master Theorem where applicable.

\subsection*{(a) MergeSort on Sorted Arrays}
1. **Recurrence Relation**:
   \[
   T(n) = 2T\left(\frac{n}{2}\right) + n - 1
   \]
   - Splitting the array takes $2T(n/2)$ comparisons.
   - Merging two sorted arrays of size $n/2$ takes $n - 1$ comparisons (since each comparison merges one element).\\
2. **Solution Using Master Theorem**:
   - Here, $a = 2$, $b = 2$, and $f(n) = n - 1 = \Theta(n)$.
   - Calculate $\log_b{a} = \log_2{2} = 1$.
   - Since $f(n) = \Theta(n)$, we use Case 2 of the Master Theorem.
   \[
   T(n) = \Theta(n \log n)
   \]

\subsection*{(b) MergeSort on Reverse-Sorted Arrays}
1. **Recurrence Relation**:
   \[
   T(n) = 2T\left(\frac{n}{2}\right) + n - 1
   \]
   - The recurrence relation is the same as for sorted arrays because MergeSort's comparison count does not depend on the order of the elements.\\
2. **Solution Using Master Theorem**:
   - Same as part (a).
   \[
   T(n) = \Theta(n \log n)
   \]

\subsection*{(c) QuickSort on Sorted Arrays}
1. **Recurrence Relation**:
   \[
   T(n) = T(1) + T(n-1) + (n-1)
   \]
   - The first partition step compares each element to the pivot, making $n-1$ comparisons.\\
   - The best pivot selection (first element in sorted array) leads to one partition of size $n-1$ and one partition of size 0.\\
2. **Solution**:
   - The recurrence simplifies to:
   \[
   T(n) = T(n-1) + n - 1
   \]
   - Unroll the recurrence:
   \[
   T(n) = T(n-1) + (n-1) = T(n-2) + (n-2) + (n-1) = \ldots = T(1) + \sum_{i=1}^{n-1} i = T(1) + \frac{n(n-1)}{2}
   \]
   - Since $T(1) = 0$ (base case), we have:
   \[
   T(n) = \Theta(n^2)
   \]

\subsection*{(d) QuickSort on Arrays of All Equal Elements}
1. **Recurrence Relation**:
   \[
   T(n) = 2T(n/2) + (n-1)
   \]
   - Each partition step still compares each element to the pivot, making $n-1$ comparisons.
   - For arrays with all equal elements, each partition splits the array into two equal parts.\\
2. **Solution Using Master Theorem**:
   - Here, $a = 2$, $b = 2$, and $f(n) = n - 1 = \Theta(n)$.
   - Calculate $\log_b{a} = \log_2{2} = 1$.
   - Since $f(n) = \Theta(n)$, we use Case 2 of the Master Theorem.
   \[
   T(n) = \Theta(n \log n)
   \]

\section*{Question 2: Optimizing QuickSort for Sorted Arrays}

To optimize QuickSort for sorted arrays, we should select the pivot such that the partitions are balanced.

\subsection*{Pivot Selection Rule}
- **Rule**: Select the median of the array as the pivot. In sorted arrays, this will divide the array into two equal parts.

\subsection*{Resulting Recurrence Relation}
1. **Recurrence Relation**:
   \[
   T(n) = 2T\left(\frac{n}{2}\right) + n - 1
   \]
   - Selecting the median as the pivot results in two partitions of size $n/2$.\\
2. **Solution Using Master Theorem**:
   - Here, $a = 2$, $b = 2$, and $f(n) = n - 1 = \Theta(n)$.
   - Calculate $\log_b{a} = \log_2{2} = 1$.
   - Since $f(n) = \Theta(n)$, we use Case 2 of the Master Theorem.
   \[
   T(n) = \Theta(n \log n)
   \]

\end{document}
