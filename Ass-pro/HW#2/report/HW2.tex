\documentclass{article}
\usepackage{amsmath}
\usepackage{hyperref}
\title{CS5720 Design Annd Analysis of Algorithms}
\author{Raja Kantheti}
\date{}

\begin{document}

\maketitle

\section*{Problem 1}

\subsection*{(a) \( x(n) = x(n-1) + 5 \) for \( n > 1 \), \( x(1) = 0 \)}

\textbf{Method: Back Substitution}

To solve this, let's write out the first few terms:
\begin{align*}
x(2) &= x(1) + 5 = 0 + 5 = 5, \\
x(3) &= x(2) + 5 = 5 + 5 = 10, \\
x(4) &= x(3) + 5 = 10 + 5 = 15, \\
&\vdots \\
x(n) &= x(n-1) + 5.
\end{align*}

We can see the pattern:
\[ x(n) = 5(n-1) \]

General solution:
\[ x(n) = 5(n-1) = 5n - 5 \]

Big Theta notation:
\[ x(n) = \Theta(n) \]

\subsection*{(b) \( x(n) = 3x(n-1) \) for \( n > 1 \), \( x(1) = 0 \)}

\textbf{Method: Back Substitution}

For \( n > 1 \):
\begin{align*}
x(2) &= 3x(1) = 3 \cdot 0 = 0, \\
x(3) &= 3x(2) = 3 \cdot 0 = 0.
\end{align*}

We can see that:
\[ x(n) = 0 \]

General solution:
\[ x(n) = 0 \]

Big Theta notation:
\[ x(n) = \Theta(1) \]

\subsection*{(c) \( x(n) = x(n-1) + n \) for \( n > 0 \), \( x(0) = 0 \)}

\textbf{Method: Back Substitution}

To solve this, let's write out the first few terms:
\begin{align*}
x(1) &= x(0) + 1 = 0 + 1 = 1, \\
x(2) &= x(1) + 2 = 1 + 2 = 3, \\
x(3) &= x(2) + 3 = 3 + 3 = 6, \\
x(4) &= x(3) + 4 = 6 + 4 = 10, \\
&\vdots \\
x(n) &= x(n-1) + n.
\end{align*}

We can see the pattern:
\[ x(n) = \frac{n(n+1)}{2} \]

General solution:
\[ x(n) = \frac{n(n+1)}{2} \]

Big Theta notation:
\[ x(n) = \Theta(n^2) \]

\subsection*{(d) \( x(n) = x \left( \frac{n}{2} \right) + n \) for \( n > 1 \), \( x(1) = 1 \) (solve for \( n = 2^k \))}

\textbf{Method: Back Substitution}

Let \( n = 2^k \).
\[ x(2^k) = x(2^{k-1}) + 2^k \]

To solve this, let's use back substitution:
\begin{align*}
x(2^k) &= x(2^{k-1}) + 2^k, \\
x(2^{k-1}) &= x(2^{k-2}) + 2^{k-1}, \\
x(2^k) &= x(2^{k-2}) + 2^{k-1} + 2^k, \\
&\vdots \\
x(2^k) &= x(1) + 2 + 4 + \ldots + 2^k.
\end{align*}

Sum of the geometric series:
\[ x(2^k) = 1 + 2 + 4 + \ldots + 2^k = 2^{k+1} - 1 \]

Since \( n = 2^k \):
\[ x(n) = 2n - 1 \]

General solution:
\[ x(n) = 2n - 1 \]

Big Theta notation:
\[ x(n) = \Theta(n) \]

\subsection*{(e) \( x(n) = x \left( \frac{n}{3} \right) + 1 \) for \( n > 1 \), \( x(1) = 1 \) (solve for \( n = 3^k \))}

\textbf{Method: Back Substitution}

Let \( n = 3^k \).
\[ x(3^k) = x(3^{k-1}) + 1 \]

To solve this, let's use back substitution:
\begin{align*}
x(3^k) &= x(3^{k-1}) + 1, \\
x(3^{k-1}) &= x(3^{k-2}) + 1, \\
x(3^k) &= x(3^{k-2}) + 1 + 1, \\
&\vdots \\
x(3^k) &= x(1) + k.
\end{align*}

Since \( x(1) = 1 \) and \( k = \log_3 n \):
\[ x(n) = 1 + \log_3 n \]

General solution:
\[ x(n) = 1 + \log_3 n \]

Big Theta notation:
\[ x(n) = \Theta(\log n) \]

\section*{Problem 2}

\paragraph*{Master Theorem: }
Solved some of this probems using a direct method called the \href{https://www.programiz.com/dsa/master-theorem}{Master theorem}.

\subsection*{(a) \( T(n) = 2T \left( \frac{n}{2} \right) + n^3 \)}

\textbf{Method: Master Theorem}

To solve this, we use the Master Theorem for divide-and-conquer recurrences:
\[ T(n) = aT \left( \frac{n}{b} \right) + f(n) \]

Here, \( a = 2 \), \( b = 2 \), and \( f(n) = n^3 \).

We compare \( f(n) \) with \( n^{\log_b a} \):
\[ \log_b a = \log_2 2 = 1 \]

Since \( f(n) = n^3 \) which is \( \Theta(n^3) \), and \( n^3 > n^1 \):
By case 3 of the Master Theorem:
\[ T(n) = \Theta(n^3) \]

General solution:
\[ T(n) = \Theta(n^3) \]

\subsection*{(b) \( T(n) = T(\sqrt{n}) \sqrt{n} + n \). Here, assume that \( T(2) = c \).}

\textbf{Method: Back Substitution}

Let \( n = 2^k \). Then \( \sqrt{n} = 2^{k/2} \).

\[ T(2^k) = T(2^{k/2}) \cdot 2^{k/2} + 2^k \]

To solve this, we use back substitution:
\begin{align*}
T(2^k) &= T(2^{k/2}) \cdot 2^{k/2} + 2^k, \\
T(2^{k/2}) &= T(2^{k/4}) \cdot 2^{k/4} + 2^{k/2}, \\
T(2^k) &= \left( T(2^{k/4}) \cdot 2^{k/4} + 2^{k/2} \right) \cdot 2^{k/2} + 2^k.
\end{align*}

Let's denote \( T(2^k) = T(2^{k/2}) \cdot 2^{k/2} \cdot 2^{k/4} \cdot 2^{k/8} \cdot \ldots + k \cdot 2^k \).

Therefore, the general form can be derived, but since it's complex, we often use asymptotic notation:
\[ T(n) = \Theta(n \log \log n) \]

General solution:
\[ T(n) = \Theta(n \log \log n) \]

\subsection*{(c) \( T(n) = 2T \left( \frac{n}{2} \right) + n \log n \)}

\textbf{Method: Master Theorem}

To solve this, we use the Master Theorem for divide-and-conquer recurrences:
\[ T(n) = aT \left( \frac{n}{b} \right) + f(n) \]

Here, \( a = 2 \), \( b = 2 \), and \( f(n) = n \log n \).

We compare \( f(n) \) with \( n^{\log_b a} \):
\[ \log_b a = \log_2 2 = 1 \]

Since \( f(n) = n \log n \) which is \( \Theta(n \log n) \), and \( n \log n \) is asymptotically equal to \( n^{\log_b a} \cdot \log^k n \) where \( k = 1 \):
By case 2 of the Master Theorem:
\[ T(n) = \Theta(n \log^2 n) \]

General solution:
\[ T(n) = \Theta(n \log^2 n) \]

\subsection*{(d) \( T(n) = 3T \left( \frac{n}{2} \right) + n \log n \)}

\textbf{Method: Master Theorem}

To solve this, we use the Master Theorem for divide-and-conquer recurrences:
\[ T(n) = aT \left( \frac{n}{b} \right) + f(n) \]

Here, \( a = 3 \), \( b = 2 \), and \( f(n) = n \log n \).

We compare \( f(n) \) with \( n^{\log_b a} \):
\[ \log_b a = \log_2 3 \approx 1.584 \]

Since \( f(n) = n \log n \) which is \( O(n^{\log_b a}) \), and \( n \log n < n^{\log_b a} \):
By case 1 of the Master Theorem:
\[ T(n) = \Theta(n^{\log_b a}) \]

General solution:
\[ T(n) = \Theta(n^{\log_2 3}) \]

\end{document}
